\documentclass[12pt]{revtex4-1}
\topmargin=-1 cm
\usepackage{graphicx}
\usepackage{epstopdf}
\usepackage{bm}
\usepackage{hyperref}
\newcommand{\be}{\begin{equation}}
\newcommand{\ee}{\end{equation}}
\newcommand{\bea}{\begin{eqnarray}}
\newcommand{\eea}{\end{eqnarray}}
\usepackage{dcolumn}
\usepackage{amsmath}
\usepackage{amssymb}
\usepackage{multirow}
\usepackage{adjustbox}
\usepackage{latexsym}
\usepackage{braket}
\usepackage{setspace}
\usepackage{subcaption}
\newtheorem{Lemma}{Lemma}
%\newtheorem{Proof}{Proof}
\newcommand{\red}{\color[rgb]{0.8,0,0}}
\newcommand{\blue}{\color[rgb]{0,0,0.6}}
\newcommand{\green}{\color[rgb]{0.0,0.7,0.0}}


% commenting commands
%\usepackage[normalem]{ulem}%This enables strike through text using the \sout{} command.
\usepackage{color}
\newcommand{\Tin}[1]{{\color{red} #1}}
\newcommand{\Tout}[1]{{\color{blue} \sout{#1}}}

\begin{document}

\begin{center}
  \textbf{Reply to Second Report of the First Referee  on the manuscript WX10023A Gupta ``Distillation of genuine tripartite Einstein-Podolsky-Rosen steering"}
\end{center}

We thank the Referee for the report and suggestions for improvement of
the presentation. 

In the following, we reproduce the Referee's report verbatim and address the Referee's comments.

{\red{\bf{Referee's Comment:}}} \textit{The authors have revised the manuscript according to the last report
and answered the concerns in a detailed way, most of them in a
satisfactory manner (see specifics down below). They further gave good
and comprehensible arguments why their manuscript includes non-trivial
results. I get and understand their argumentation, even though I think
most of the results were expected the authors are right with their
reasoning that the asymmetry introduced by the tripartite structure
could have introduced additional complications. My main point of
critique was that all results could be obtained by a straightforward
generalization of the protocols in ref. [39] and the fact that the
steering distillation used in the manuscript is actually based on the
mathematical equality to entanglement distillation in this specific
case. I am still not fully convinced that the way the authors handled
this issue is the optimal one. I would like to suggest that they
specifically mention this property when the specific cases are studied
(for instance page 5 paragraph A). This gives the inexperienced reader
more insight and does not leave the more experienced reader with the
feeling that it should be properly addressed instead of just
mentioning it in the conclusion. 
}

{\blue{\bf{Author's Response:}}} We thank the Referee for acknowledging our arguments 
about the non-trivial nature of our results.  In order to address the issue of optimality of the protocol properly, one has to perform a resource theoretic analysis of 
genuine tripartite EPR  steering distillation protocol, which is byond the scope of the present study. In view of the Referees' comments, in the revised manuscript we have added a new paragraph in the introduction, and also the subsection II C to highlight
the advantage of designing a genuine tripartite EPR steering distillation protocol. Here, we have also mentioned how our genuine tripartite steering distillation strategy can be used 
to distill the genuine tripartite entanglement with lesser resource, whereas, genuine tripartite entanglement distillation protocol cannot always be used for distilling genuine tripartite EPR steering. \\
 
{\red{\bf{Referee's Comment:}}} \textit{Apart from this, I think the manuscript can be published in PRA, due
to the improved version and the good reasoning of the authors.}

{\blue{\bf{Author's Response:}}} We thank the Referee again for the above comment, and for recommending our manuscript for publication.\\

{\red{\bf{Referee's Specific Comment:}}} \textit{Maybe it is better to change "We show that the perfectly genuine
tripartite EPR steerable assemblage..." in the abstract to "We show that a perfectly genuine tripartite EPR steerable
assemblage..." since the notion of perfectly genuine tripartite EPR is
by no means unique (as discussed in the manuscript and the response to
my comments).}

{\blue{\bf{Author's Response:}}} We have changed `the' to `a' as suggested by Referee.\\

{\red{\bf{Referee's Specific Comment:}}} \textit{Maybe change the notation from $M_{a \vert X_x}$ to $M_{a \vert x}$
like in ref. [24] since this notation is much more common and the
additional "X" does not add any additional information.}

{\blue{\bf{Author's Response:}}} We have changed the notation $X_x$ to $x$ as suggested by Referee in the entire manuscript.\\

{\red{\bf{Referee's Specific Comment:}}} \textit{There are two typos on p. 18 above eq. (G1) and under eq (G1) in
the brackets of the assemblage.}

{\blue{\bf{Author's Response:}}} We have corrected the typos in the revised manuscript. \\
%\begin{figure}
%\begin{subfigure}{8.5cm}
%  \centering
%  \includegraphics[width=7.5cm]{Eq10GGHZ.pdf}
  %\caption{\tiny{Eq.10}}
  
%\end{subfigure}%
%\begin{subfigure}{8.5cm}
%  \centering
%  \includegraphics[width=7.5cm]{Eq11.pdf}
 % \caption{\tiny{Variation of the minimum number of copies ($N_{\text{min}}$)  with the state parameter $\theta$ required to achieve near perfect assemblage fidelity ($\mathcal{F}_{A} \sim O(1) - 10^{-5}$) in our 1SDI scenario  for GGHZ states.}}
%\end{subfigure}
%\caption{Variation of the genuine steering inequality versus the state parameter $\theta$.}
%\label{fig1}
%\end{figure}

%\begin{figure}
%\begin{subfigure}{8.5cm}
  %\centering
  %\includegraphics[width=7.5cm]{Eq12c03.pdf}
  %\caption{$c_0 = 0.3$}
  
%\end{subfigure}%
%\begin{subfigure}{8.5cm}
%  \centering
%  \includegraphics[width=7.5cm]{Eq13.pdf}
 % \caption{\tiny{Variation of the minimum number of copies ($N_{\text{min}}$)  with the state parameter $\theta$ required to achieve near perfect assemblage fidelity ($\mathcal{F}_{A} \sim O(1) - 10^{-5}$) in our 1SDI scenario  for GGHZ states.}}
%\end{subfigure}
%\caption{Variation of the genuine steering inequality versus the state parameter of W class state.}
%\label{fig2}
%\end{figure}
\newpage 

\begin{center}
  \textbf{Reply to  Report of the Second Referee on the manuscript WX10023A Gupta ''Distillation of genuine tripartite Einstein-Podolsky-Rosen steering"}
\end{center}

We thank the Referee for the report and suggestions for improvement of
the presentation. 

In the following, we reproduce the Referee's report verbatim and address the Referee's comments.

{\red{\bf{Referee's Comment:}}} \textit{In the manuscript, the Authors have provided a protocol to distill
genuine ‘perfect’ three-qubit steerable assemblages from partially
steerable assemblages in cases of both one and two untrusted parties.
Specifically, they have taken the assemblages corresponding to the GHZ
and W states respectively as the ‘perfect’ assemblages for GHZ-class
and W-class states, and then provided distillation protocols starting
form generalized GHZ and generalized W states both in asymptotic and
in finite copy level. I believe that the manuscript is mathematically
sound and well written. Moreover, I find Author’s reply to the
comments of Referee A is very much adequate. Therefore, I would like
to recommend the Manuscript for publication in PRA after minor
modifications.
}

{\blue{\bf{Author's Response:}}} We thank the Referee for carefully reviewing the manuscript and our response to first Referee's report. 
We appreciate that the Referee has found our work mathematically sound and properly written. We further thank the Referee for recommending
our manuscript for publication.
\\

{\red{\bf{Referee's Comment:}}} \textit{Referee A is correct in pointing out that the definition of a
‘perfect’ assemblage in three-qubit setting is ambiguous. In their
reply, Author’s have correctly pointed out that there are two LOCC
inequivalent class of states, namely GHZ and W class. Therefore, it
might be justified to take assemblages that ‘are derived from the GHZ
state and W state by applying orthogonal Von Neumann measurements of
rank-1’. I think the question of this ambiguity and Author’s reasoning
for such choice must be clarified more clearly, specifically in the
Introduction, Sec. IIA1 and in Sec. III.}

{\blue{\bf{Author's Response:}}} We thank the Referee for this suggestion as it gives us an opportunity to improve presentation of the manuscript.
Based on the Referee's suggestion, we have  discussed the idea of perfect assemblage and why it could be scenario dependent in the Introduction.
In Sec. IIA1, we have provided an  argument for the choice of the target assemblage (perfect assemblage) for the specific scenario. See also the  discussion  presented in Sec. III (below Eq.(25); on page 9, 1st column, last paragraph; on page 10, 1st and 2nd column; on page 11, 2nd column).\\

{\red{\bf{Referee's Comment:}}} \textit{Although the results presented are logical next-step to Ref. [39], I
am very much satisfied with the Authors reply about the uniqueness of
their results in tripartite settings. However, I would suggest the
Authors to put these reasoning in Sec. IV so that the readers can
understand the uniqueness of the present work in a much better way.
They should also put a discussion about steering distillation vs.
entanglement distillation in the present context (preferably as a
subsection of Sec. II).
}

{\blue{\bf{Author's Response:}}} In view of the Referee's comments, we have added a subsection -- Sec. IIC -- where we have discussed the difference between genuine tripartite steering and genuine tripartite entanglement distillation,  why 
designing a genuine tripartite steering distillation protocol is useful, and how it can be used to distill the genuine tripartite entanglement with lesser resource, whereas genuine tripartite entanglement
distillation protocol may not always be used for distilling genuine tripartite steering. Based on the Referee's suggestion, we have now mentioned the non-trivial results of the present work in Sec. IV.\\

{\red{\bf{Referee's Comment:}}} \textit{Several times in the text, Authors interchangeably used (GHZ-class
states / generalized GHZ states) and also (W-class states /
generalized W states), although they are not necessarily the same. For
eg. Eq. (35) defines generalized W states and not the whole W-class
states. They should be more careful towards the nomenclature.}

{\blue{\bf{Author's Response:}}} We thank the Referee for pointing this out. In this work we have used only generalized GHZ and generalized W states and not the entire GHZ and W class states. 
We have made the necessary changes in the relevant sentences of the revised manuscript.\\


\clearpage
\begin{center}
   \textbf{List of changes}
\end{center}

The changes in the manuscript in response to the suggestions of the Referees
are mentioned below. (The changes in the text have been indicated in Blue color
in the revised manuscrtipt).
\begin{itemize}
%\item Minor changes in the abstract.
\item In the abstract `the' has been replaced to `a' (Page 1).
\item A few sentences have been added to justify scenario dependent perfect assemblage for steering  (Page 2, Column 1).
\item Notations $X_y$, $Y_y$ and $Z_z$ have been updated as $x, y$ and $z$   for better readability.
\item Some statements have been added to justify our choice of perfect assemblage in the 1SDI and 2SDI scenario (Page 3-4, Section IIA1).
\item A subsection ``Steering vs entanglement distillation" has been added (Page 5, Column 1).
\item A discussion has been added to mention the non-trivial results of the work (Page 13-14, Section IV).
\item Typos above and under Eq. G1 have been corrected.
\item Captions of the figures are modified to mention that the quantities plotted have no units.
\item Figures have been updated (font of axes labels increased).
%\item Few additions in the Conclusion section (Page 11).
%\item Reference 16 added.


\end{itemize}




\end{document} 